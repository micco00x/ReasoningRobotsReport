\section{Atari Breakout}
This section contains the main part of the project, describing in detail our
starting point, how the program has been developed, the results achieved and the
comparison with the original implementation, which uses a non-Atari version
of the game Breakout \cite{DBLP:journals/corr/abs-1807-06333} with the same
number of bricks used in the Atari implementation of \texttt{gym},
already introduced in section \ref{sec:openaigym}.

Initially, the non-Atari version of Breakout (built on PyGame) is introduced,
its implementation is discussed and the results on the training with a brick
matrix of dimension 6$\times$18 are presented. Then the \texttt{gym}
environment \texttt{BreakoutNoFrameskip-v4} is presented and compared with the
PyGame breakout. A detailed description of the implementation of the project
is described, discussing in detail how the features have been extracted from
the environment (both the robot and the goal features), how temporal goals
have been used to evaluate the state of the bricks and how everything is
connected together in order to make it work with \texttt{gym}. In the end,
the main experiments performed on \texttt{BreakoutNoFrameskip-v4} are
presented.

\subsection{PyGame Breakout}
As introduced above, in \cite{DBLP:journals/corr/abs-1807-06333} a PyGame
version of the Breakout game has been used in order to test the algorithms
introduced in the paper. The implementation of this Breakout easily allows
to determine the state of the environment, saving the status of the bricks
(either in the scene or broken), the position of the ball, the direction of
the ball and the position of the paddle. This makes it possible to reduce
a lot the computational time of the implementation of the agent since the
data it receives are already preprocessed in order to have a complete
overview of the environment, allowing to focus on higher-level reasoning
tasks.

Originally, the paper focused on a Breakout environment with a brick matrix
of dimension 4$\times$5. Since the Atari version of Breakout is dealing with
a brick matrix of dimension 6$\times$18, a new test has been performed to
make the two environments comparable, in order to better understand the
potentiality of $\text{LTL}_f/\text{LDL}_f$, the use of non-Markovian rewards
and how to approach complex \texttt{gym} environments in the future.

The method managed to correctly break all the bricks in around one hour of
training. TODO: add three figures with initial state, middle, final.
TODO: add plot of rewards (shall this go to Experiments subsec?).

\subsection{Arcade Learning Environment}
This works aims at comparing the \texttt{gym} environment
\texttt{BreakoutNoFrameskip-v4}, already introduced in section
\label{sec:openaigym} with the non-Atari Breakout used in
\cite{DBLP:journals/corr/abs-1807-06333}. In last years, the reinforcement
learning community has grown a lot thanks to the introduction of \texttt{gym}
and deep reinforcement learning algorithms \cite{mnih2015humanlevel} that
managed to easily solve complex games that are considered difficult also
for humans, often achieving better results than expert human gamers. More
and more algorithms are introduced every year, exploiting GPU resources
and managing to solve harder games like Montezuma Revenge \cite{uber-goexplore}.
The popularity of deep reinforcement learning begun with the introduction of
Arcade Learning Environment (ALE) that includes most
famous arcade Atari games \cite{bellemare13arcade}.

The main characteristics of \texttt{gym} (or ALE) environments is that the
world can be observed only from the pixels of the screen, putting the
algorithms at the same level of the human, that can only observe the display
while playing. This makes the game a lot more complex since a more abstract
reasoning strategy is needed in order to solve the game. This hypothesis should
make \texttt{gym} games a lot harder than the non-Atari version of Breakout
that has been used to test algorithms that work with non-Markovian rewards.

\subsection{Implementation}
Intro to implementation.

\subsubsection{Robot Features Extractor}
\texttt{RobotFeatureExtractor} (OpenCV). Extracts features of the robot (robot
and ball positions).
\lstinputlisting[caption=Robot feature extractor Python implementation.,
    language=Python]{implementation/breakoutfull-breakoutrobotfeatureextractor.py}

\subsubsection{Goal Features Extractor}
\texttt{GoalFeatureExtractor} (OpenCV). Extracts 6x18 table representation
of the bricks in order to evaluate a formula.
\lstinputlisting[caption=Goal feature extractor Python implementation.,
    language=Python]{implementation/breakoutfull-breakoutgoalfeatureextractor.py}

\texttt{*Ext} used to improve implementation.

\subsubsection{Temporal Goals}
$\text{LTL}_f/\text{LDL}_f$ implementation (with Marco Favorito libraries).
\lstinputlisting[caption=$\text{LTL}_f/\text{LDL}_f$ formulas Python implementation.,
    language=Python]{implementation/breakoutfull-breakoutcompleterowstemporalevaluator.py}

Atari wrappers (OpenAI).
