\section{Atari Breakout}
\label{section:ataribreakout}
This section contains the main part of the project, describing in detail our
starting point, how the program has been developed, the results achieved and the
comparison with the original implementation, which uses a non-Atari version
of the game Breakout \cite{DBLP:journals/corr/abs-1807-06333} with the same
number of bricks used in the Atari implementation of \texttt{gym},
already introduced in section \ref{sec:openaigym}.

Initially, the non-Atari version of Breakout (built on PyGame) is introduced,
its implementation is discussed and the results on the training with a brick
matrix of dimension 6$\times$18 are presented. Then the \texttt{gym}
environment \texttt{BreakoutNoFrameskip-v4} is presented and compared with the
PyGame breakout. A detailed description of the implementation of the project
is described, discussing in detail how the features have been extracted from
the environment (both the robot and the goal features), how temporal goals
have been used to evaluate the state of the bricks and how everything is
connected together in order to make it work with \texttt{gym}.

\subsection{PyGame Breakout}
As introduced above, in \cite{DBLP:journals/corr/abs-1807-06333} a PyGame
version of the Breakout game has been used in order to test the algorithms
introduced in the paper. The implementation of this Breakout easily allows
to determine the state of the environment, saving the status of the bricks
(either in the scene or broken), the position of the ball, the direction of
the ball and the position of the paddle. This makes it possible to reduce
a lot the computational time of the implementation of the agent since the
data it receives are already preprocessed in order to have a complete
overview of the environment, allowing to focus on higher-level reasoning
tasks.

Originally, the paper focused on a Breakout environment with a brick matrix
of dimension 4$\times$5. Since the Atari version of Breakout is dealing with
a brick matrix of dimension 6$\times$18, a new test has been performed to
make the two environments comparable, in order to better understand the
potentiality of $\text{LTL}_f\text{/LDL}_f$, the use of non-Markovian rewards
and how to approach complex \texttt{gym} environments in the future.

\subsection{Arcade Learning Environment}
This work aims at comparing the \texttt{gym} environment
\texttt{BreakoutNoFrameskip-v4}, already introduced in section
\ref{sec:openaigym} with the non-Atari Breakout used in
\cite{DBLP:journals/corr/abs-1807-06333}. In last years, the reinforcement
learning community has grown a lot thanks to the introduction of \texttt{gym}
and deep reinforcement learning algorithms \cite{mnih2015humanlevel} that
managed to easily solve complex games that are considered difficult also
for humans, often achieving better results than expert human gamers. More
and more algorithms are introduced every year, exploiting GPU resources
and managing to solve harder games like Montezuma Revenge \cite{uber-goexplore}.
The popularity of deep reinforcement learning begun with the introduction of
Arcade Learning Environment (ALE) that includes most
famous arcade Atari games \cite{bellemare13arcade}.

The main characteristics of \texttt{gym} (or ALE) environments is that the
world can be observed only from the pixels of the screen, putting the
algorithms at the same level of the human, that can only observe the display
while playing. This makes the game a lot more complex since a more abstract
reasoning strategy is needed in order to solve the game. This hypothesis should
make \texttt{gym} games a lot harder than the non-Atari version of Breakout
that has been used to test algorithms that work with non-Markovian rewards.

\subsubsection{Atari Wrappers}
Before introducing the main part of the implementation it is important to
discuss the use of Atari wrappers, introduced with the OpenAI baselines
\cite{openai-baselines}, that simplify the interaction with the environment,
lightening the code and managing important aspects of the game.
The wrappers used in the project, as it is possible to notice from Algorithm
\ref{lst:run-atariwrappers-py}, are the following:
\begin{itemize}
    \item \texttt{EpisodicLifeEnv}: make an ``end of life'' be the end
        of the episode resetting the environment only on the true
        game over, this helps in value estimation;
    \item \texttt{FireResetEnv}: use ``Fire'' as starting action in order
        to launch the ball;
    \item \texttt{MaxAndSkipEnv}: returns only the skip-th frame, this
        reduces the amount of frame the agent has to deal with (set to 4
        in the main experiments).
\end{itemize}

Note that line 1 defines the \texttt{gym} environment used in the project,
namely \texttt{BreakoutNoFrameskip-v4}. All the wrappers extend the class
\texttt{gym.Wrapper}.

\lstinputlisting[caption={Initialization of the Atari Breakout environment
    and the use of Atari wrappers from OpenAI.},
    label={lst:run-atariwrappers-py},
    language=Python]{implementation/breakoutfull-run-atariwrappers.py}

\subsection{Implementation}
Following the pipeline introduced in \cite{DBLP:journals/corr/abs-1807-06333}
and described in Fig. \ref{fig:rl-temporalgoals-pipeline}, the robot features
extractor, the goal features extractor and the temporal goals are described.
In particular, their implementation makes use of OpenCV \cite{opencv-library}
in order to deal
with images easily and a Python implementation of FLLOAT \cite{python-flloat}
to deal with $\text{LTL}_f\text{/LDL}_f$ formulas. Note that the abstract
class \texttt{TemporalEvaluator} in Algorithm \ref{lst:temporal-goals-py}
makes use of the libraries Pythomata \cite{pythomata} and RLTG \cite{rltg} to
build automata from the desired $\text{LTL}_f\text{/LDL}_f$ formulas in order
to make it possible to receive non-Markovian rewards.

\subsubsection{Robot Features Extractor}
The implementation of the robot features extractor is shown in Algorithm
\ref{lst:robot-features-extractor-py}. The class extends the abstract class
\texttt{BreakoutRobotFeatureExtractor}, which has been developed just to
keep a consistent structure among other implementation of other robot features
extractors. The class implements two methods: the constructor
\texttt{__init__}, which takes as input a \texttt{gym} object defining the space
of the observation, and the method \texttt{_extract}, which takes as input
the observation \texttt{input} coming from the environment and a dictionary
\texttt{kwargs} containing other optional parameters.

The constructor \texttt{__init__} defines the robot features space (lines 4-7)
with two \texttt{gym.spaces.Discrete} objects defined with values 287 and
157, respectively the maximum possible values (extreme excluded) that the
position of the ball can have with respect to the paddle and the height of the
ball. Then, internal representation of the position of the ball and of the
paddle are initialized (lines 9-12). The boolean \texttt{self.still_image}
is used to avoid repetitions of the same observation in the method
\texttt{_extract} since the same observation is considered twice. In the end,
the superconstructor is called (line 14) in order to finalize the
construction of the object.

The method \texttt{\_extract} checks weather the image has been already seen
or not (lines 17-19) has explained above. Then, the position of the paddle
is extracted from the observation (an image) on lines 20-35. Initially,
only the bottom part of the image is extracted from the observation (line 21),
it is then converted to a gray-scale image (line 22) so that it is possible
to apply a threshold function in order to make the paddle white and the
rest of the image black (line 23). In this way it is possible to find contours
of the objects contained in that part of the image (there should be only
one actually since only the bottom part is considered) and extract the centroid
of the paddle, in this way the variable
\texttt{paddleX} is updated. Similarly, the position of the ball is extracted
(lines 37-69) from the upper part of the image. Here, it is important to
actually check that the centroid is part of the ball since there could be
objects that are part of the bricks. Fortunately the ball has a unique RGB color
$(200, 72, 72)$ which simplifies this step.

Finally, the internal representation of the object is updated (lines 71-73)
and a tuple containing data specified in the constructor is returned (line 75).
\lstinputlisting[caption=Robot feature extractor Python implementation.,
    label={lst:robot-features-extractor-py},
    language=Python]{implementation/breakoutfull-breakoutrobotfeatureextractor.py}

\subsubsection{Goal Features Extractor}
The implementation of the goal features extractor is shown in Algorithm
\ref{lst:goal-features-extractor-py}. The class extends the abstract class
\texttt{FeatureExtractor}, which has been developed just to
keep a consistent structure among other implementation of other goal features
extractors. The class implements two methods: the constructor
\texttt{__init__}, which takes as input a \texttt{gym} object defining the space
of the observation and the number of rows and columns composing the bricks
matrix, and the method \texttt{_extract}, which takes as input
the observation \texttt{input} coming from the environment and a dictionary
\texttt{kwargs} containing other optional parameters.

The constructor \texttt{__init__} saves the number of rows and columns
(lines 3-4) of the
bricks matrix used in its internal representation and defines the space used
by the method \texttt{_extract} to return objects (a simple representation
of the bricks matrix). In the end, the superconstructor is called (line 6)
in order to finalize the construction of the object.

The method \texttt{\_extract} simply returns a numpy representation of the
bricks matrix seen from the observation, which is in turn given by the environment
each time the agent takes an action. The algorithm cycles on the pixels
of the image representing the observation checking weather they are black
or not. In particular, it is possible to determine this by checking the two
pixels on the upper left and upper right part of the brick comparing their
color (lines 10-20) with the background (black). If both of them are black, then the brick
has already been destroyed, otherwise it is still present in the environment.
Note that it is necessary to check both pixels because one of the two could
be different due to the presence of the ball during the game. As mentioned
above, the method return a 6$\times$18 numpy matrix representing the status of
the bricks (each element will have value 0 if the brick has been destroyed,
1 otherwise).
\lstinputlisting[caption=Goal feature extractor Python implementation.,
    label={lst:goal-features-extractor-py},
    language=Python]{implementation/breakoutfull-breakoutgoalfeatureextractor.py}

\subsubsection{Temporal Goals}
The implementation of temporal goals is shown in Algorithm
\ref{lst:temporal-goals-py}. It is divided in three parts:
\begin{itemize}
    \item \texttt{get_breakout_lines_formula}: a function that determines the
        $\text{LTL}_f\text{/LDL}_f$ formula as a string, in order to be
        later parsed by the FLLOAT parser;
    \item \texttt{BreakoutCompleteLinesTemporalEvaluator}: a
        temporal evaluator class that handles rows and columns as lines;
    \item \texttt{BreakoutCompleteRowsTemporalEvaluator}: a temporal evaluator
        class that extends \texttt{BreakoutCompleteLinesTemporalEvaluator} and
        work on rows, this is the main class used in the project regarding
        temporal goals.
\end{itemize}

In particular, \texttt{get_breakout_lines_formula} (lines 1-15) generates a
$\text{LTL}_f\text{/LDL}_f$ formula as explained in Section
\ref{section:nonmarkovianrewards}, but extended to a general case of a Breakout
consisting of a bricks matrix of size $n \times m$.

\texttt{BreakoutCompleteLinesTemporalEvaluator} extends the RLTG abstract class
\texttt{TemporalEvaluator} which contains the abstract method
\texttt{fromFeaturesToPropositional}. Its constructor \texttt{__init__} (lines
20-38) parses the $\text{LTL}_f\text{/LDL}_f$ specified by the function
\texttt{get_breakout_lines_formula} (lines 22-30) in order to pass it to
the superconstructor (lines 33-38) that will manage the construction of the
automata using Pythomata and RLTG libraries. The method
\texttt{fromFeaturesToPropositional}, inherited from \texttt{TemporalEvaluator},
maps the bricks matrix to a propositional formula in order to update the
automata while training the agent.

\texttt{BreakoutCompleteRowsTemporalEvaluator} has the same structure of the
previous class with the exception of the boolean \texttt{self.bottom_up} that
specifies weather breaking the rows from top to bottom or viceversa.
For all the experiments of the project this variable has been set to
\texttt{False} in order to incite the agent to find a strategy that makes the
ball go to the upper part of the environment in order to break the bricks
without making any effort.
\lstinputlisting[caption=$\text{LTL}_f\text{/LDL}_f$ formulas Python implementation.,
    label={lst:temporal-goals-py},
    language=Python]{implementation/breakoutfull-breakoutcompleterowstemporalevaluator.py}
