%++++++++++++++++++++++++++++++++++++++++
% Don't modify this section unless you know what you're doing!
%\documentclass[letterpaper,12pt]{article}
\documentclass[a4paper]{article}
\usepackage{tabularx} % extra features for tabular environment
\usepackage{amsmath}  % improve math presentation
\usepackage{graphicx} % takes care of graphic including machinery
\usepackage{subcaption} %  for subfigures environments 
%\usepackage[margin=1in,letterpaper]{geometry} % decreases margins
\usepackage{cite} % takes care of citations
%\usepackage[final]{hyperref} % adds hyper links inside the generated pdf file
%\hypersetup{
%	colorlinks=true,       % false: boxed links; true: colored links
%	linkcolor=blue,        % color of internal links
%	citecolor=blue,        % color of links to bibliography
%	filecolor=magenta,     % color of file links
%	urlcolor=blue
%}
%++++++++++++++++++++++++++++++++++++++++
\usepackage{indentfirst}
\usepackage{tensor}
\usepackage{amssymb}
\allowdisplaybreaks
\usepackage{bm}
\newcommand{\at}[2][]{#1|_{#2}}
\newcommand\numberthis{\addtocounter{equation}{1}\tag{\theequation}}
\newcommand\norm[1]{\left\lVert#1\right\rVert}

% Python code, taken from:
% https://www.quora.com/What-is-the-optimal-way-to-include-Python-code-in-a-LaTeX-document
\usepackage{color}
\usepackage{listings}
\usepackage{setspace}
\definecolor{Code}{rgb}{0,0,0}
\definecolor{Decorators}{rgb}{0.5,0.5,0.5}
\definecolor{Numbers}{rgb}{0.5,0,0}
\definecolor{MatchingBrackets}{rgb}{0.25,0.5,0.5}
\definecolor{Keywords}{rgb}{0,0,1}
\definecolor{self}{rgb}{0,0,0}
\definecolor{Strings}{rgb}{0,0.63,0}
\definecolor{Comments}{rgb}{0,0.63,1}
\definecolor{Backquotes}{rgb}{0,0,0}
\definecolor{Classname}{rgb}{0,0,0}
\definecolor{FunctionName}{rgb}{0,0,0}
\definecolor{Operators}{rgb}{0,0,0}
\definecolor{Background}{rgb}{0.98,0.98,0.98}
\lstset{
    breaklines=true,
    postbreak=\mbox{\textcolor{red}{$\hookrightarrow$}\space}
}
\lstdefinelanguage{Python}{
numbers=left,
numberstyle=\footnotesize,
numbersep=1em,
xleftmargin=1em,
framextopmargin=2em,
framexbottommargin=2em,
showspaces=false,
showtabs=false,
showstringspaces=false,
frame=l,
tabsize=4,
% Basic
basicstyle=\ttfamily\small\setstretch{1},
backgroundcolor=\color{Background},
% Comments
commentstyle=\color{Comments}\slshape,
% Strings
stringstyle=\color{Strings},
morecomment=[s][\color{Strings}]{"""}{"""},
morecomment=[s][\color{Strings}]{'''}{'''},
% keywords
morekeywords={import,from,class,def,for,while,if,is,in,elif,else,not,and,or,print,break,continue,return,True,False,None,access,as,,del,except,exec,finally,global,import,lambda,pass,print,raise,try,assert},
keywordstyle={\color{Keywords}\bfseries},
% additional keywords
morekeywords={[2]@invariant,pylab,numpy,np,scipy},
keywordstyle={[2]\color{Decorators}\slshape},
emph={self},
emphstyle={\color{self}\slshape},
%
}
\linespread{1.3}
% Python code [end].

\usepackage{caption}
\captionsetup[lstlisting]{font={small}}
\renewcommand{\lstlistingname}{Algorithm}

% Fontsize of figure smaller than normalsize:
\captionsetup[figure]{font=small}
\captionsetup[table]{font=small}

\usepackage{breqn}

\begin{document}

\title{Atari Breakout with\\$\text{LTL}_f\text{/LDL}_f$ Goals}
%\author{Ivan Bergonzani, Michele Cipriano, Armando Nania}
%\date{\today}
%\maketitle


\makeatletter
\let\thetitle\@title
\let\theauthor\@author
\let\thedate\@date
\makeatother

\begin{titlepage}
	\centering
    \vspace*{0.5 cm}
    \includegraphics[scale = 0.75]{images/SapienzaLogo}\\[1.0 cm]	% University Logo

    \vspace*{-0.3cm}
    \textsc{\large Department of Computer, Control and\\Management Engineering}\\[2.0 cm]	% Department Name
    \vspace*{1.2cm}

    { \fontsize{20.74pt}{18.5pt}\selectfont\bfseries \thetitle \par } % title

    \vspace*{0.1cm}
    \textsc{\Large Elective in Artificial Intelligence:\\Reasoning Robots}\\[0.5 cm] % course name

    \vspace*{2.8cm}
	\begin{minipage}{0.4\textwidth}
		\begin{flushleft} \large
			\emph{Professor:}\\
			Giuseppe De Giacomo\\
            %Computer Science Department\\
		\end{flushleft}
	\end{minipage}~
	\begin{minipage}{0.4\textwidth}
		\begin{flushright} \large
			\emph{Students:} \\
            Ivan Bergonzani \\
			Michele Cipriano\\
            Armando Nania
            %Semester\\
		\end{flushright}

	\end{minipage}\\[2 cm]

    \vspace{0.2cm}
    \rule{\linewidth}{0.2 mm} \\[0.3 cm]
    \vspace*{-0.3cm}
    Academic Year 2017/2018
\end{titlepage}

\tableofcontents
\newpage


%\begin{abstract}
%In this experiment we studied a very important physical effect by measuring the
%dependence of a quantity $V$ of the quantity $X$ for two different sample
%temperatures.  Our experimental measurements confirmed the quadratic dependence
%$V = kX^2$ predicted by Someone's first law. The value of the %mystery parameter
%$k = 15.4\pm 0.5$~s was extracted from the fit. This value is
%not consistent with the theoretically predicted $k_{theory}=17.34$~s. We attribute this
%discrepancy to low efficiency of our $V$-detector.
%\end{abstract}

\section{Introduction}
The aim of the project is to extend the work introduced in
\cite{DBLP:journals/corr/abs-1807-06333} testing the algorithms on a much
harder environment, namely an Atari version of Breakout available in the
\texttt{gym} framework.

The report is structured in the following way. Section \ref{section:rl}
introduces the basic theory behind reinforcement learning, briefly describing
the Q-Learning and the SARSA algorithms, which have been used to train the
agent in the experiments. Section \ref{section:nonmarkovianrewards}
describes $\text{LTL}_f\text{/LDL}_f$ non-Markovian rewards and how they can
be used to train a RL agent. Section \ref{sec:openaigym} describes the
framework \texttt{gym} from OpenAI, which provides a level of abstraction
on Arcade Learning Environment (ALE), which in turn provides a huge amount
of classical Atari games. Section \ref{section:ataribreakout} discusses about
the main implementation of the project, comparing the PyGame version of the
Breakout game of the paper with one of the Atari Breakout versions of
\texttt{gym}. Atari wrappers are introduced and robot features extractor,
goal features extractor and temporal goals used in the implementation are
presented in details. Section \ref{subsec:experiments} presents all the
experiments performed during the development of the project, highlighting
differences between the two environments discussed in the previous section
and all the changes applied to both environments in order to better
understand weaknesses from both parts. The report ends with the
conclusion the summarizes the work done and discusses about possible
future works.

\clearpage

\section{Reinforcement Learning}
\label{section:rl}
Reinforcement learning \cite{Suttonrl18} is an area of machine learning
which aims at studying how to develop agents that can interact with
their environment maximing a cumulative reward. The environment can be
formally defined as a Markov Decision Process (MDP), which is a tuple
$\langle S, A, \delta, R \rangle$ where $S$ is a finite set of states that can represent
the environment, $A$ is a finite set of actions that can be perform by
an agent in the environment, $\delta$ is a probability function modeling
the transition from a state to another when performing a certain action and
$R$ is a reward function which models the reward received by the environment
when performing a certain action which makes the agent move from a state
to another.

An interesting property of the MDP is that it satisfies the Markov property,
hence, future states that will be reached by the agent do not depend
on the past interaction of the environment, but just on the current state.
This makes it possible to define the transition and the reward function
depending only on the current state (and of course the action and the future
state of interest).

This section considers two common reinforcement learning algorithm,
namely Q-Learning and SARSA, which have been used in our experiments
in order to train an agent interacting with an Atari Breakout environment
(section \ref{subsec:experiments}).

\subsection{Q-Learning}
Q-Learning is a temporal difference (TD) algorithm that directly approximates
the optimal action-value
function. This method guarantees to find an optimal behaviour under the
assumption that all the state-action pairs are updated infinitely many times. It is
defined \cite{Suttonrl18} by the following equation:
\begin{equation}
    \label{eq:qlearning-update-function}
    Q(S_t, A_t) \leftarrow Q(S_t, A_t) + \alpha \Big[ R_{t+1} +
        \gamma \max_{a} Q(S_{t+1}, a) - Q(S_t, A_t) \Big]
\end{equation}

Let's briefly discuss the implementation used in our project by studying
the Python implementation (Algorithm \ref{lst:qlearning-py}). The algorithm
is defined by the class \texttt{QLearning} that extends the abstract class
\texttt{TDBrain}. The constructor of the class (lines 2-4) simply calls its
parent constructor that will initialize the parameters of the object, hence,
the observation space and the action space (\texttt{gym} objects), the strategy
used by the policy function, which is $\varepsilon$-greedy by default and the
hyperparameters $\gamma$, $\alpha$ and $\lambda$ of the upper class.
The abstract method inherited from \texttt{TDBrain} is \texttt{update\_Q},
which should be implemented in order to define how to update the action-state
table. The method (lines 6-21) simply follows Eq.
\ref{eq:qlearning-update-function}.
\lstinputlisting[caption=Q-Learning algorithm Python implementation.,
    label={lst:qlearning-py},
    language=Python]{implementation/TDBrain-QLearning.py}

\subsection{SARSA}
A similar TD algorithm is the SARSA algorithm, which name
comes from the fact that at each timestep a quintuple $\langle S_t, A_t,
R_{t+1}, S_{t+1}, A_{t+1} \rangle$ is considered. As before, SARSA converges
to an optimal action-value function under the assumption that all state-action
pairs are updated infinitely many times. It is defined
\cite{Suttonrl18} by the following equation:
\begin{equation}
    \label{eq:sarsa-update-equation}
    Q(S_t, A_t) \leftarrow Q(S_t, A_t) + \alpha \Big[ R_{t+1} +
        \gamma Q(S_{t+1}, A_{t+1}) - Q(S_t, A_t) \Big]
\end{equation}

Let's briefly discuss the implementation used in our project by studying the
implementation (Algorithm \ref{lst:sarsa-py}), as done before with the
Q-Learning algorithm. Again, the algorithm is defined by the class
\texttt{Sarsa} that extends the abstract class \texttt{TDBrain}. The constructor
of the class (lines 2-4) calls its parent constructor initializing the
parameters of the upper class exactly in the same way as the class
\texttt{QLearning}. The class implements the inherited method
\texttt{update\_Q} by following Eq. \ref{eq:sarsa-update-equation} (lines
6-17).
\lstinputlisting[caption=SARSA algorithm Python implementation.,
    label={lst:sarsa-py},
    language=Python]{implementation/TDBrain-Sarsa.py}

\clearpage

\section{$\text{LTL}_f/\text{LDL}_f$ Non-Markovian Rewards}
Recently, non-Markovian reward decision processes (NMRDPs)
has attracted interest in the scientific community because of the possibility
of specifying them as MDPs with $\text{LTL}_f/\text{LDL}_f$ non-Markovian
rewards \cite{DBLP:journals/corr/abs-1807-06333}. In particular, it is possible
to model the problem with two separate representations of the world, one for
the agent (low-level) and one for the goal (expressed in terms of high-level
fluents).

This section presents the approach used in
\cite{DBLP:journals/corr/abs-1807-06333}, where an efficient method has been
developed in order to work with NMRDPs. The theory behind the main idea is
quickly described and an example on a theoretical Breakout environment is
discussed in order to be used in the following sections easily.

\subsection{Theoretical Background}
Before describing the problem, let's give a formal definition of NMRDP.
A non-Markovian reward decision process is a tuple $M = \langle S, A, \delta,
\bar{R} \rangle$, with $S$ finite set of states that can represent the
environment, $A$ is a finite set of actions that can be performed by an agent
in the environment, $\delta$ is a probability function modeling
the transition from a state to another when performing a certain action and
$\bar{R}: (S \times A)^* \rightarrow \mathbb{R}$ is a function from
finite state-action sequences (traces) to real-values that represents the
reward given by the environment when performing a certain state-action
sequence. Specifying a non-Markovian reward function explicitely is
difficult even when considering a finite number of traces. Luckily, the
$\text{LTL}_f/\text{LDL}_f$ formalism allows to specify $\bar{R}$
implicitely using a set of pairs $\{ (\phi_i, r_i) \}_{i=1}^m$ with
$\phi_i$ boolean
proposition over the components of the state vector and $r_i$ such that,
given a current trace $\pi = \langle s_, a_1, \dots, s_{n-1}, a_n \rangle$,
the agent receives at $s_n$ a reward $r_i$ if $\phi_i$ is satisfied by $\pi$,
hence:
\begin{equation}
    \bar{R}(\pi) =
        \begin{cases}
            r_i & \text{if } \pi \vDash \phi_i \\
            0 & \text{otherwise}
        \end{cases}
\end{equation}

Since the NMRDP rewards are based on traces, instead of state-action pairs,
typical learning algorithms like Q-learning or SARSA cannot be used.
Nevertheless, it has been shown \cite{DBLP:journals/corr/abs-1807-06333} that
for any NMRDP $M = \langle S, A, \delta, \{ (\phi_i, r_i) \}_{i=1}^m \rangle$
there exists an MDP $M' = \langle S', A', \delta' R' \rangle$ that is equivalent
to $M$. The idea behind the proof consists in starting from an initial
decision process $M_{ag}^{goal} = \langle S, A, R, \mathcal{L},
\delta_{ag}^g, \{ (\phi_i, r_i) \}_{i=1}^m \rangle$ with
$\text{LTL}_f/\text{LDL}_f$ goals (with $\mathcal{L}$ set of of configuration
of the high-level features needed for expressing $\phi_i$),
transform it into a NMRDP in order to
further transform it into a MDP where it is possible to execute learning
algorithms such as Q-learning. All the details are out of the scope of the
project and are discussed in \cite{DBLP:journals/corr/abs-1807-06333}. The
set of states $S$ is used to express low-level features of the agent.
\begin{figure}
    \centering
    \includegraphics[width=0.85\textwidth]{images/rl-temporalgoals-pipeline.png}
    \caption{TODO: description.}
    \label{fig:rl-temporalgoals-pipeline}
\end{figure}

\subsection{Examples}
How it can be used to train a RL model.

\clearpage

\section{OpenAI Gym}
\label{sec:openaigym}
OpenAI \texttt{gym} \cite{1606.01540} is a toolkit for developing and comparing
reinforcement learning algorithms, without making assumptions about the
structure of the agent interacting with the environment, in order to
keep development flexible to updates on both sides.

\subsection{Framework}
The framework of \texttt{gym} allows to interact easily
with an environment, giving to developers the tools they need to perform
actions and to observe the state of the environment itself. In this way it is
possible to focus more on the development of the agent without spending
time on the structure of the world.

\texttt{gym} makes it possible to interact with multiple kinds of environments.
Among these, the authors of the framework developed the support for
Arcade Learning Environment \cite{bellemare13arcade}, which includes all the
classing Atari games, including Breakout, which has been used in this project.

\subsection{Examples}
Let's consider a simple example to understand how \texttt{gym} works and
how the framework can be used to interact with an environment.
The description will follow Algorithm \ref{lst:gym-breakout-example-py}.

\lstinputlisting[caption={Example of a random interaction with the \texttt{gym}
    environment \texttt{BreakoutNoFrameskip-v4}, used also in our experiments
    of subsection \ref{subsec:experiments}.},
    label={lst:gym-breakout-example-py},
    language=Python]{implementation/gym-breakout-example.py}

Initially (line 1) the framework is imported. Then (line 3-4) an environment
is created specifying its name and initializing it. The program makes a
random agent interact randomly with the environment for 1000 timesteps (lines
6-11) before closing the environment. Line 7 renders the current
observation of the environment on screen, line 8-9 performs a random action
between those available in this Brekout version, note that the method
\texttt{step} return an \texttt{observation} (shown in Fig.
\ref{fig:gym-breakout-image-example}), which is an array of pixels
that represent the current state of the environment, a \texttt{reward},
which is a value return by the game after performing the specified action
\texttt{action}, a boolean value \texttt{done}, which is \texttt{True} when
the game is over, \texttt{False} otherwise, and \texttt{info} which contains
extra information about the game. Lines 10-11 handles the case when the game
is over, resetting the environment.

\begin{figure}
    \centering
    \includegraphics[width=0.3\textwidth]{images/gym-breakout-image-example.jpg}
    \caption{Observation of a frame of the environment
        \texttt{BreakoutNoFrameskip-v4}.}
    \label{fig:gym-breakout-image-example}
\end{figure}

\clearpage

\section{Atari Breakout}
This section contains the main part of the project, describing in detail our
starting point, how the program has been developed, the results achieved and the
comparison with the original implementation, which uses a non-Atari version
of the game Breakout \cite{DBLP:journals/corr/abs-1807-06333} with the same
number of bricks used in the Atari implementation of \texttt{gym},
already introduced in section \ref{sec:openaigym}.

Initially, the non-Atari version of Breakout (built on PyGame) is introduced,
its implementation is discussed and the results on the training with a brick
matrix of dimension 6$\times$18 are presented. Then the \texttt{gym}
environment \texttt{BreakoutNoFrameskip-v4} is presented and compared with the
PyGame breakout. A detailed description of the implementation of the project
is described, discussing in detail how the features have been extracted from
the environment (both the robot and the goal features), how temporal goals
have been used to evaluate the state of the bricks and how everything is
connected together in order to make it work with \texttt{gym}. In the end,
the main experiments performed on \texttt{BreakoutNoFrameskip-v4} are
presented.

\subsection{PyGame Breakout}
As introduced above, in \cite{DBLP:journals/corr/abs-1807-06333} a PyGame
version of the Breakout game has been used in order to test the algorithms
introduced in the paper. The implementation of this Breakout easily allows
to determine the state of the environment, saving the status of the bricks
(either in the scene or broken), the position of the ball, the direction of
the ball and the position of the paddle. This makes it possible to reduce
a lot the computational time of the implementation of the agent since the
data it receives are already preprocessed in order to have a complete
overview of the environment, allowing to focus on higher-level reasoning
tasks.

Originally, the paper focused on a Breakout environment with a brick matrix
of dimension 4$\times$5. Since the Atari version of Breakout is dealing with
a brick matrix of dimension 6$\times$18, a new test has been performed to
make the two environments comparable, in order to better understand the
potentiality of $\text{LTL}_f/\text{LDL}_f$, the use of non-Markovian rewards
and how to approach complex \texttt{gym} environments in the future.

The method managed to correctly break all the bricks in around one hour of
training. TODO: add three figures with initial state, middle, final.
TODO: add plot of rewards (shall this go to Experiments subsec?).

\subsection{Arcade Learning Environment}
This works aims at comparing the \texttt{gym} environment
\texttt{BreakoutNoFrameskip-v4}, already introduced in section
\label{sec:openaigym} with the non-Atari Breakout used in
\cite{DBLP:journals/corr/abs-1807-06333}. In last years, the reinforcement
learning community has grown a lot thanks to the introduction of \texttt{gym}
and deep reinforcement learning algorithms \cite{mnih2015humanlevel} that
managed to easily solve complex games that are considered difficult also
for humans, often achieving better results than expert human gamers. More
and more algorithms are introduced every year, exploiting GPU resources
and managing to solve harder games like Montezuma Revenge \cite{uber-goexplore}.
The popularity of deep reinforcement learning begun with the introduction of
Arcade Learning Environment (ALE) that includes most
famous arcade Atari games \cite{bellemare13arcade}.

The main characteristics of \texttt{gym} (or ALE) environments is that the
world can be observed only from the pixels of the screen, putting the
algorithms at the same level of the human, that can only observe the display
while playing. This makes the game a lot more complex since a more abstract
reasoning strategy is needed in order to solve the game. This hypothesis should
make \texttt{gym} games a lot harder than the non-Atari version of Breakout
that has been used to test algorithms that work with non-Markovian rewards.

\subsection{Implementation}
Intro to implementation.

\subsubsection{Robot Features Extractor}
\texttt{RobotFeatureExtractor} (OpenCV). Extracts features of the robot (robot
and ball positions).
\lstinputlisting[caption=Robot feature extractor Python implementation.,
    language=Python]{implementation/breakoutfull-breakoutrobotfeatureextractor.py}

\subsubsection{Goal Features Extractor}
\texttt{GoalFeatureExtractor} (OpenCV). Extracts 6x18 table representation
of the bricks in order to evaluate a formula.
\lstinputlisting[caption=Goal feature extractor Python implementation.,
    language=Python]{implementation/breakoutfull-breakoutgoalfeatureextractor.py}

\texttt{*Ext} used to improve implementation.

\subsubsection{Temporal Goals}
$\text{LTL}_f/\text{LDL}_f$ implementation (with Marco Favorito libraries).
\lstinputlisting[caption=$\text{LTL}_f/\text{LDL}_f$ formulas Python implementation.,
    language=Python]{implementation/breakoutfull-breakoutcompleterowstemporalevaluator.py}

Atari wrappers (OpenAI).

\subsection{Experiments}
\label{subsec:experiments}
Results with 6x18 non-ATARI Breakout (+CODE).

Results with our experiments (+CODE).

\clearpage

\section{Experiments}
\label{subsec:experiments}
Results with 6x18 non-ATARI Breakout (+CODE).

Results with our experiments (+CODE).

\clearpage

\section{Conclusion}
Why it does not work.

Summary + differences between the two environments.

Future works (neural networks and parallel computation).

\clearpage

%++++++++++++++++++++++++++++++++++++++++
% References section will be created automatically
% with inclusion of "thebibliography" environment
% as it shown below. See text starting with line
% \begin{thebibliography}{99}
% Note: with this approach it is YOUR responsibility to put them in order
% of appearance.

\bibliography{bibliography}
\bibliographystyle{ieeetr}

%\begin{thebibliography}{99}

%\bibitem{melissinos}
%A.~C. Melissinos and J. Napolitano, \textit{Experiments in Modern Physics},
%(Academic Press, New York, 2003).

%\bibitem{Cyr}
%N.\ Cyr, M.\ T$\hat{e}$tu, and M.\ Breton,
% "All-optical microwave frequency standard: a proposal,"
%IEEE Trans.\ Instrum.\ Meas.\ \textbf{42}, 640 (1993).

%\bibitem{Wiki} \emph{Expected value},  available at
%\texttt{http://en.wikipedia.org/wiki/Expected\_value}.

%\end{thebibliography}


\end{document}
