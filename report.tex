%++++++++++++++++++++++++++++++++++++++++
% Don't modify this section unless you know what you're doing!
%\documentclass[letterpaper,12pt]{article}
\documentclass[a4paper]{article}
\usepackage{tabularx} % extra features for tabular environment
\usepackage{amsmath}  % improve math presentation
\usepackage{graphicx} % takes care of graphic including machinery
%\usepackage[margin=1in,letterpaper]{geometry} % decreases margins
\usepackage{cite} % takes care of citations
%\usepackage[final]{hyperref} % adds hyper links inside the generated pdf file
%\hypersetup{
%	colorlinks=true,       % false: boxed links; true: colored links
%	linkcolor=blue,        % color of internal links
%	citecolor=blue,        % color of links to bibliography
%	filecolor=magenta,     % color of file links
%	urlcolor=blue
%}
%++++++++++++++++++++++++++++++++++++++++
\usepackage{indentfirst}
\usepackage{tensor}
\usepackage{amssymb}
\allowdisplaybreaks
\usepackage{bm}
\newcommand{\at}[2][]{#1|_{#2}}
\newcommand\numberthis{\addtocounter{equation}{1}\tag{\theequation}}
\newcommand\norm[1]{\left\lVert#1\right\rVert}

\begin{document}

\title{Atari Breakout with $\text{LTL}_f/\text{LDL}_f$ Goals\\(draft)}
\author{Ivan Bergonzani, Michele Cipriano, Armando Nania}
\date{\today}
\maketitle

%\begin{abstract}
%In this experiment we studied a very important physical effect by measuring the
%dependence of a quantity $V$ of the quantity $X$ for two different sample
%temperatures.  Our experimental measurements confirmed the quadratic dependence
%$V = kX^2$ predicted by Someone's first law. The value of the %mystery parameter
%$k = 15.4\pm 0.5$~s was extracted from the fit. This value is
%not consistent with the theoretically predicted $k_{theory}=17.34$~s. We attribute this
%discrepancy to low efficiency of our $V$-detector.
%\end{abstract}


\section{Introduction}

General overview of the project, what has been implemented,
etc.

\section{Conclusion}
Summary of the project, possible future developments and
conclusions.

%++++++++++++++++++++++++++++++++++++++++
% References section will be created automatically
% with inclusion of "thebibliography" environment
% as it shown below. See text starting with line
% \begin{thebibliography}{99}
% Note: with this approach it is YOUR responsibility to put them in order
% of appearance.

\bibliography{bibliography}
\bibliographystyle{ieeetr}

%\begin{thebibliography}{99}

%\bibitem{melissinos}
%A.~C. Melissinos and J. Napolitano, \textit{Experiments in Modern Physics},
%(Academic Press, New York, 2003).

%\bibitem{Cyr}
%N.\ Cyr, M.\ T$\hat{e}$tu, and M.\ Breton,
% "All-optical microwave frequency standard: a proposal,"
%IEEE Trans.\ Instrum.\ Meas.\ \textbf{42}, 640 (1993).

%\bibitem{Wiki} \emph{Expected value},  available at
%\texttt{http://en.wikipedia.org/wiki/Expected\_value}.

%\end{thebibliography}


\end{document}
