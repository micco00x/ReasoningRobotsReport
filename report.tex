%++++++++++++++++++++++++++++++++++++++++
% Don't modify this section unless you know what you're doing!
%\documentclass[letterpaper,12pt]{article}
\documentclass[a4paper]{article}
\usepackage{tabularx} % extra features for tabular environment
\usepackage{amsmath}  % improve math presentation
\usepackage{graphicx} % takes care of graphic including machinery
%\usepackage[margin=1in,letterpaper]{geometry} % decreases margins
\usepackage{cite} % takes care of citations
%\usepackage[final]{hyperref} % adds hyper links inside the generated pdf file
%\hypersetup{
%	colorlinks=true,       % false: boxed links; true: colored links
%	linkcolor=blue,        % color of internal links
%	citecolor=blue,        % color of links to bibliography
%	filecolor=magenta,     % color of file links
%	urlcolor=blue
%}
%++++++++++++++++++++++++++++++++++++++++
\usepackage{indentfirst}
\usepackage{tensor}
\usepackage{amssymb}
\allowdisplaybreaks
\usepackage{bm}
\newcommand{\at}[2][]{#1|_{#2}}
\newcommand\numberthis{\addtocounter{equation}{1}\tag{\theequation}}
\newcommand\norm[1]{\left\lVert#1\right\rVert}

\begin{document}

\title{Atari Breakout with\\$\text{LTL}_f/\text{LDL}_f$ Goals}
%\author{Ivan Bergonzani, Michele Cipriano, Armando Nania}
%\date{\today}
%\maketitle


\makeatletter
\let\thetitle\@title
\let\theauthor\@author
\let\thedate\@date
\makeatother

\begin{titlepage}
	\centering
    \vspace*{0.5 cm}
    \includegraphics[scale = 0.75]{images/SapienzaLogo}\\[1.0 cm]	% University Logo

    \vspace*{-0.3cm}
    \textsc{\large Department of Computer, Control and\\Management Engineering}\\[2.0 cm]	% Department Name
    \vspace*{1.2cm}

    { \fontsize{20.74pt}{18.5pt}\selectfont\bfseries \thetitle \par } % title

    \vspace*{0.1cm}
    \textsc{\Large Elective in Artificial Intelligence:\\Reasoning Robots}\\[0.5 cm] % course name

    \vspace*{2.8cm}
	\begin{minipage}{0.4\textwidth}
		\begin{flushleft} \large
			\emph{Professor:}\\
			Giuseppe De Giacomo\\
            %Computer Science Department\\
		\end{flushleft}
	\end{minipage}~
	\begin{minipage}{0.4\textwidth}
		\begin{flushright} \large
			\emph{Students:} \\
            Ivan Bergonzani \\
			Michele Cipriano\\
            Armando Nania
            %Semester\\
		\end{flushright}

	\end{minipage}\\[2 cm]

    \vspace{0.2cm}
    \rule{\linewidth}{0.2 mm} \\[0.3 cm]
    \vspace*{-0.3cm}
    Academic Year 2017/2018
\end{titlepage}

\tableofcontents
\newpage


%\begin{abstract}
%In this experiment we studied a very important physical effect by measuring the
%dependence of a quantity $V$ of the quantity $X$ for two different sample
%temperatures.  Our experimental measurements confirmed the quadratic dependence
%$V = kX^2$ predicted by Someone's first law. The value of the %mystery parameter
%$k = 15.4\pm 0.5$~s was extracted from the fit. This value is
%not consistent with the theoretically predicted $k_{theory}=17.34$~s. We attribute this
%discrepancy to low efficiency of our $V$-detector.
%\end{abstract}


\section{Introduction}
Introduction to the whole project, structure of the report and summary of
the work.

\clearpage
\section{Reinforcement Learning}
Introduction to RL.

\subsection{Q-Learning}
Q-Learning algorithm.

\subsection{SARSA}
SARSA algorithm.

\clearpage
\section{$\text{LTL}_f/\text{LDL}_f$ Non-Markovian Rewards}
Introduction to the research paper and how can it be used to train
a RL model.

\clearpage
\section{OpenAI Gym}
Introduction to the framework.

Examples.

\clearpage
\section{ATARI Breakout}
Original implementation of the paper (non-ATARI).

ATARI Breakout and differences from the other one.

Results with 6x18 non-ATARI Breakout (+CODE).

Results with our experiments (+CODE).

\texttt{RobotFeatureExtractor} (OpenCV).

\texttt{GoalFeatureExtractor} (OpenCV).

\texttt{*Ext} used to improve implementation.

$\text{LTL}_f/\text{LDL}_f$ implementation (with Marco Favorito libraries).

\clearpage
\section{Conclusion}
Why it does not work.

Summary + differences between the two environments.

Future works (neural networks and parallel computation).

%++++++++++++++++++++++++++++++++++++++++
% References section will be created automatically
% with inclusion of "thebibliography" environment
% as it shown below. See text starting with line
% \begin{thebibliography}{99}
% Note: with this approach it is YOUR responsibility to put them in order
% of appearance.

\clearpage
\bibliography{bibliography}
\bibliographystyle{ieeetr}

%\begin{thebibliography}{99}

%\bibitem{melissinos}
%A.~C. Melissinos and J. Napolitano, \textit{Experiments in Modern Physics},
%(Academic Press, New York, 2003).

%\bibitem{Cyr}
%N.\ Cyr, M.\ T$\hat{e}$tu, and M.\ Breton,
% "All-optical microwave frequency standard: a proposal,"
%IEEE Trans.\ Instrum.\ Meas.\ \textbf{42}, 640 (1993).

%\bibitem{Wiki} \emph{Expected value},  available at
%\texttt{http://en.wikipedia.org/wiki/Expected\_value}.

%\end{thebibliography}


\end{document}
